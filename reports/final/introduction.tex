\chapter{Introduction and Background}
\label{ch:introduction}
The EduCraft project is concerned with developing a series of
extensions\footnote{These are commonly called `mods'. Throughout the report,
we will refer to our product as `the mod'.} for the popular computer game
Minecraft, aimed at promoting collaborative learning in primary schools,
particularly with reference to numeracy education. In this first section, some
of the advantages of collaborative learning are explored in contrast to
conventional teaching methods, and the game of Minecraft is introduced in the
context of education.

\section{Collaborative learning}
\begin{quote}
``Group interaction allows students to negotiate meanings, to express
themselves in the language of the subject and to establish a more intimate
and dialectical contact with academic and teaching staff than formal
methods permit.'' \cite[p.~1]{jacques00}
\end{quote}
Group work has long been recognised as a key component in any form of
successful education---it enables students to become more involved in their
own work, and to engage with it in a different manner from what is normally
seen in classroms. Johnson and Johnson observe that ``The human species seems
to have a \textit{cooperation imperative}'', and that cooperation plays
a central role in most areas of life \cite[p.~12]{johnson94}.

Collaborative or cooperative learning is contrasted with two other `goal
structures': competitive learning, and individualistic learning. Competitive
learning consists of activities which set the students tasks which some
are expected to `win', by getting the highest score or completing the activity
in the shortest time, for example. Individualistic learning is learning where
there is no interaction between students at all---no comparisons are made,
and students are solely concerned with working by themselves.

Johnson and Johnson argue that each of these goal structures has a role
to play in education, but that each is suitable for different
purposes \cite{johnson94}.  Collaborative learning, specifically, is to
be used in situations where the material being taught is especially
complex or conceptual. This makes it ideal for use in maths lessons:
Yackel, \textit{et al.} describe the benefits of group work in teaching maths
to second-grade\footnote{Roughly equivalent to year 5 (ages 7--8) in the UK}
students in the US \cite{yackel91}.

\section{Minecraft}
\begin{quote}
``Minecraft is a game about breaking and placing blocks. At first, people
built structures to protect against nocturnal monsters, but as the game grew
players worked together to create wonderful, imaginative things.''
\cite{website:minecraft}
\end{quote}

Minecraft was initially developed in 2009, as a game where players could
explore a randomly-generated world and place blocks to build structures. From
the outset, Minecraft was not designed as a game that could be `won'---in
game industry terms, it was intended to be a `sandbox' game where players set
their own goals.

There have been a number of articles written exploring the potential use of
Minecraft in supporting education \cite{brand13}, \cite{short12}. Habgood has
already established the usefulness of computer games in general in education
\cite{habgood07}; the question is whether Minecraft can be used in the ways
that he suggests. This is what EduCraft seeks to establish, at least tentatively.

The official Minecraft Wiki has its own page dedicated to describing possible ways
of using Minecraft in education \cite{website:mcwiki}, and suggests using the
game's in-built system of building items:
\begin{quote}
    ``The crafting system can help in teaching basic math (e.g. I need 3 Sugar
    Cane for Paper), which transitions to multiplication (I need 3 Paper and 1
    Leather for a Book, and 3 Books for a Bookshelf, so I need 9 Paper and 3 Leather
    all together) and division (When I create Paper I get 3 at once, so 9/3 =
    3 times per Bookshelf I'll have to create Paper).''
\end{quote}

We intend to build something more overtly mathematical than this basic concept,
and the remainder of the report describes the requirements that have been set
for the project, along with a record of our initial design and prototypes.
