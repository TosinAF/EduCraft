\chapter{Testing and Analysis}
\label{ch:testing}

\section{Introduction}
Every project requires testing to ensure that it meets its stated objectives.
This chapter describes the methods that were adopted to test the EduCraft mod
`in the field'---a description of the unit tests performed on the code itself
is given in Appendix \ref{apdx:unit-tests}.


\section{Game Measures}
Before describing the tests we performed, we first give a brief outline of the
measures that we developed in order to give some framework for assessing the success
of the game.

The measures we developed were based on the work of Nicola Whitton \cite{whitton07},
and were gathered into three categories: engagement, collaboration, and system.
These were designed to measure the children's engagement with the game, the level
of collaboration they displayed, and the integrity of the system respectively.
A full descripion of the measures is presented in Appendix \ref{apdx:measures}, so
we omit a full discussion here.

The measures were used as guides to the observations we carried out in the following
tests, in order to help focus our attention on the key things.


\section{Pilot Study}
The first test we performed was with a simple level containing a single problem. The
purpose of this test was to see how quickly a child who had never played Minecraft
before could learn the game and work out how to use the mod; as such, this test
involved only a single player, an 11 year-old boy. He played through the level
twice, with the option to ask for as much help as he needed from the observers.

\subsection{Observations}
On his first attempt, John's\footnote{All children's names have been changed.} unfamiliarity
with the controls proved to be a rather large obstacle. He was uncertain how to move
the character around effectively, and without knowing how Minecraft mechanics like
crafting tables usually worked, he was unable to easily learn how to make the mathematical
operators required. However, on the second play through he was much more adept---having
learned the controls, he was more able to focus on the objectives of the level.

Despite his increased engagement with the game in general, John's engagement with the
level was not quite so high. Once he had learnt the controls he was more interested
in heading off to try and do his own thing rather than following the set course of
the level. He missed many of the signs and messages we had set out to provide
instruction for the player, and so the observers had to verbally guide him through.

\subsection{Conclusions}
We drew three key conclusions from the pilot study, the first of which provided
general knowledge of how children learn to play Minecraft, whilst the last two gave
particular insight into how to design levels more effectively.

\subsubsection{Learning Minecraft}
From John's initial inexperience, we determined that it would be unrealistic to
expect any children who were new to Minecraft to try and learn both the basic
mechanics of the game and how to use our mod at the same time. To use the mod
in schools, children would either need separate time to learn how to play,
or would already need to be familiar with the controls.

\subsubsection{Attention to instructions}
John's lack of attention to the signs and messages placed throughout the level
was problematic. Although the use of signs to describe the level is an accepted
standard amongst older players, John simply didn't read any of the messages. This
knowledge was fed into our design of the final game, since we realised
that to be successful we would require one of the testers to give verbal instructions
to the children to guide them through\footnote{See Chapter \ref{ch:design} for
a discussion of our approaches in game design.}.

\subsubsection{Attention to route}
John was also more interested in exploring the level for himself rather than following
the path we had planned. This was facilitated by the very open nature of the level;
there was nothing forcing John to advance as expected. This was also fed into the
design of the final level---we realised we had to limit the children's options, so
as to force them to follow the correct path\footnote{Again, see Chapter \ref{ch:design}.}.


\section{Firbeck Academy}
The second test was much more comprehensive, and involved a day-long visit to
a local primary school. We brought the completed game\footnote{Described in Chapters
\ref{ch:design} and \ref{ch:implementation}.}, and had intended to have four
children play through at a time in a place where we could observe their gameplay.

\subsection{Approach}
The actual day ran somewhat differently from what we had planned for. Instead of
having a single group of four children to play through the game, we were introduced
to the entire school, and learned that the day would involve each year group visiting
various projects including ours. Instead of having the focussed attention of a group
of four for the full 30--40 minutes we had planned, we had larger groups of children
for a duration of approximately 10 minutes each.

The children still played through the level, with different groups taking over from
each other. The two testers were still taking care to observe the groups' progress,
and offered help to the children where necessary.

\subsection{Observations}
The children's engagement with the game varied dramatically from group to group
and from child to child. Some payed very close attention to the game and appeared
interested in completing the objectives, whilst others were more concerned with
messing around and trying to break the level. Larger groups nearly always displayed
lower levels of engagement, whilst the smaller groups tended to engage more. Smaller
groups were also much more likely to pay attention to the instructions given
by the testers, and to follow them. As with the pilot study, players who were
already familiar with the game's controls engaged more than those who were unfamiliar.

There was also a lack of collaboration, shown most fully in a lack of progress through the level:
no group managed to complete the `introductory' area, and very few made it through
the first area. The children were communicating with one another, but not about the
task at hand. However, the groups who payed more attention
to the testers did manage to achieve some level of collaboration when they were
prompted to work together: one stage required the children to place three numbers
in ascending order, and when the tester talked the children through the task---\textit{``Okay,
who has collected any numbers? Which numbers are they? Okay, which do you
think is the smallest? Put it in the slot\ldots''}---they did work together and talk
to each other about the concepts at hand.

\subsection{Conclusions}
The findings of the test at Firbeck Academy share some similarities with those
from the pilot study, though there were also a number of additional insights we
gained from assessing the children's performance on the day.

\subsubsection{Learning environment}
One of the biggest things we found was that the environment we were testing in
vastly hindered our efforts to get the children to play through the level as expected---
having their friends sat behind them trying to advise or take over distracted the
players, and the children were generally rather excited by the change in their
routine and so were distracted anyway. The large amounts of noise in the classroom,
generated by the other projects, made it difficult to get a small group to
focus on the instructions of the tester. As noted above, the smaller groups were
able to engage and progress more, which was probably helped by the fact that there
were fewer distractions.

\subsubsection{Need for guidance}
Similar to the pilot test, the players at Firbeck Academy appeared to prefer to try
and do their own thing rather than read the signs and follow the path of the game.
This had been compensated for in the design of the level by constructing it in
such a way as to prevent them from going places they weren't meant to, but they
still required a lot of guidance from the testers to actually make progress.

The solution that was settled on after the first test was to have a fifth player
in the game, controller by one of the testers, who would guide the other players
in-game. This helped greatly, as it made it much easier for the players to
see where they were meant to go, as they had someone in the game that they could follow.


\section{Discussion}
In this section, we discuss the lessons learned from the testing process according
to the three categories of game measures\footnote{See Appendix \ref{apdx:measures}.},
and then complete the chapter with some general remarks.

\subsection{Engagement}
Once the children learned the controls, all those who played the game were engaged
quite well in one way or another. Although the children were not focussed
explicitly on the learning objectives of the level, they all enjoyed playing Minecraft.
We believe that the surrounding environment (i.e. the levels of noise and the number
of players) was the single largest factor that impaced on engagement: those groups
who experienced fewer distractions made much more progress, and it is felt that
were these distractions removed entirely the players would engage much more fully.

With reference to the day of testing at Firbeck, the children were not expecting to
be learing maths---they were having an `ICT day'---but it is felt that if the
game were set in the context of a proper maths lesson, the players would be more
committed to paying attention to the maths content. As it was, they were simply
enjoying playing on computers.

\subsection{Collabration}
In the test at Firbeck, collaboration was seriously limited by the difficulty of
communicating with the other players meaningfully in a noisy classroom. Our
observations appear to indicate that the children required prompting to work
together, rather than it being a spontaneous action, although this may be because
time constraints prevented us from fully explaining the aims of the game to
the children.

When prompted, the children were able to work together, and we believe that given
time to test in a suitable environment where the players were not distracted
by friends or noise, a greater level of collaboration would be possible.

\subsection{General remarks}
The day at Firbeck shows that there is great potential for Minecraft to be used
in education, provided that it is used in an appropriate environment. Evidence
from testing would indicate that, at least to begin with, the children playing
would require supervision and guidance from someone who knew how the game
worked, which might limit its application in regular classes with limited
numbers of teaching staff, but it is felt that given further testing EduCraft
could be developed into an incredibly useful tool for teaching maths.
