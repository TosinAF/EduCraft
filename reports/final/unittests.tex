\chapter{Unit Testing}
\label{apdx:unit-tests}

\section{Approach}
Unit testing is considered a vital part of any software engineering
project, as it allows the team to assert that the individual components
of the system function as intended \cite{sommerville11}. They also
provide a convenient means of carrying out regression testing: when
changes are made, the same tests can be re-run, theoretically with the
same outcome.

From the outset, the EduCraft developers were concerned with how to
effectively test the mod, particularly key elements of it such as the
calculator block.

\section{JUnit and Minecraft Forge}
JUnit is a simple unit testing framework for the Java language \cite{website:junit},
and as the Eclipse IDE has built-in support for JUnit, it was the first choice
of testing tools to use.

However, on beginning to write unit tests for the mod, we encountered
a number of problems. A number of the classes comprising the core of
the Minecraft game cannot be instantiated unless there is an underlying
game client running, and this includes several key classes used in
our mod\footnote{\texttt{net.minecraft.item.ItemStack} is a prime
example.}.

This limitation meant that very few meaningful test cases could be written,
since the tests requiring these uninstantiable classes would not
compile. It was the view of the developers that, although unit testing is
considered to be best practice in software engineering \cite{sommerville11},
an alternative testing strategy would have to be devised for the project.

\section{In-Game Testing}
The approach decided upon was to set up a very basic `Test World'
with no terrain and no entities apart from the player. New blocks would
be placed in this world, and their functionality tested in-game.
Although such an approach means that testing documentation is
not available, the entire development team agreed that this was the
best solution available, given the limitations of the API.
