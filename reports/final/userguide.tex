\chapter{User Guide}
\label{apdx:user-guide}

This appendix presents a brief guide to using EduCraft. More extensive
documentation, including videos, is available at \texttt{TosinAF.github.io/EduCraft}.
URLs to the relevant videos are included in the sections below.

\section{Controls}
EduCraft uses the same controls and follows the same basic principles of regular Minecraft,
and so anyone with a basic understanding of Minecraft should be able to play Educraft. Even
without experience, the controls and concepts can be picked up quickly, and the main level
design we have produced incorporates a tutorial section at the start.

A full description of Minecraft's default controls is available on the official
Wiki page \cite{website:minecraft-controls}.


\section{Tutorial walkthrough}
This section provides a summary walkthrough of the first level of the game,
which is used as an introduction to EduCraft's new mechanics.

\subsection{First room}
The first step in the tutorial is to collect logs, which can be turned into planks and
then sticks for crafting purposes\footnote{Turning logs into sticks: \texttt{www.youtube.com/watch?v=Q3aMJ5jwxuM}}.
Sticks are important as they are what the player needs to craft operators. This is done by creating
the pattern of the desired operator in the Operator Bench, which can be used in a similar
way to a crafting table in regular Minecraft.
Players need to make patterns +, -, /, or X for plus, minus, divide, and times
\footnote{Creating operators: \texttt{www.youtube.com/watch?v=Mrg\_mkqP-D8}}.
The final step is to unlock doors by throwing the appropriate operator into the hopper
for the adjacent door\footnote{Throwing items into hoppers: \texttt{www.youtube.com/watch?v=0a-KPj\_FB80}},
and then throwing the keys found inside into
the hoppers at the end of the corridor.

\subsection{Second room}
The second room introduces players to the concept of killing enemy characters
in order to collect numbers\footnote{Killing monsters: \texttt{https://www.youtube.com/watch?v=ZswqlwMR3f4}}.
The second room requires the player to collect three numbers and then place them in
an Ordering Bench, in order to obtain a key to unlock the second
door\footnote{Using the ordering bench: \texttt{https://www.youtube.com/watch?v=MQTVB4etTfE}}.
The ordering bench in the second room will accept all numbers; benches later in the game
will accept either odd or even numbers only.

\subsection{Third room}
The final room works in a similar fashion to the second, except that the players
must use the Calculator provided to produce a target number
\footnote{Using the calculator: \texttt{https://www.youtube.com/watch?v=uxbQ4CySm30}},
rather than using and ordering bench to gain a key.


\section{Collaboration}
Educraft requires collaboration between teams of players in order for progress to be made.
One way this is done is by the necessity for players to pass resources (operators, numbers)
to each other by using rail carts. The player places the item(s) they want to send to the other
team into the chest on the cart, and then presses the button to send the cart
\footnote{Using a minecart to exchange items: \texttt{https://www.youtube.com/watch?v=k-GD8Vr8fhc}}.
