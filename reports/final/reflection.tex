\chapter{Reflections}
\label{ch:reflections}
Reflection is a very important part of a successful software engineering project
\cite{dowling00}. Looking back on a process you
have gone through and considering what went well, what did not and what could have been
done differently is a very worthwhile experience as it gives you the wisdom to improve
yourself and your work in the future. Done well, reflection can give an objective evaluation
of the success of a project which is unmistakably important.

\section{Communication}
Communication and group work was generally very good, although by no means perfect. We
used online resources and websites to communicate and coordinate work, which was useful
for keeping everyone up to date outside of our face-to-face meetings. We held formal
meetings most weeks with our supervisor, Peter Blanchfield, in order to discuss progress,
goals, obstacles, and to get advice on various aspects of the project. These were
supplemented by informal meetings with the group alone, which helped team coordination
and setting out a plan.

The importance of these meetings cannot be overstated, as they
were what kept the ball rolling throughout, and ensured decisions were made, work got done
in time and documentation of this all was kept so as to have a permanent record to refer to
at all stages of the project. As such, these meetings were very useful, although did underline
issues we had with group work; this was due to lack of involvement and communication by some
members of the group at certain stages of the project. If group dynamics had been better,
then we could have achieved more; for example, we had plans for an iOS app that never came
to fruition due to lack of time.

Furthermore, we could have implemented more mathematical
concepts to challenge older players for whom the puzzles and concepts covered in our
current mod are trivial. Just as importantly were issues we had with players abusing the
environment and playing the game in a non-productive way, which required level design decisions
to minimise. With better group dynamics and more collective time, we could have dealt with
the problems faced a lot sooner, allowing us to implement more of the desired features and
meet more of the initial requirements we set.

\section{Programming and technical issues}
Progress on the mod was sometimes slowed down by technical issues, for example in our
implementation of custom Minecraft items and blocks such as the numbers and the
Calculator Bench. However, we dealt with these obstacles well, because when our initial
ideas did not work or were implemented in a flawed manner, we would take a step back and
research the problem and consider available technology and tools in order to look for
workarounds. Using git for version control was crucial as we could try out some code,
and if it did not solve our issue or had unexpected consequences then we could simply roll
back to a previous version and start research and work on a new or modified solution.

Pairs programming was also an extremely helpful method for coding, using the combined
knowledge of two people made writing competent, useful and functional code that much
easier, and was immensely important in dealing with bugs, and tackling obstacles of
implementation. All things considered, our approach to technical issues and the process of
searching for a solution was difficult and time-consuming, but ultimately worth it as
we often developed simple, clever and effective solutions.

\section{Conclusions}
Overall, the project was largely a success. Despite limitations and issues faced, we have
created a playable mod that fulfils the expectations of our original project description:
``The project would aim at developing mods for the Minecraft game in which players could
only win by collaborating with others''. Testing the game ourselves, with individual
children and as well at the school showed promise that collaborative learning in Minecraft
is possible, even though there were a lot of problems and shortcomings, as outlined in
Chapter \ref{ch:testing}. Most of the people involved enjoyed the process and were excited
about the project's potential, and we believe that EduCraft demonstrates the usefulness
Minecraft can have as an educational tool for collaborative learning, and are looking
forward to further development.
