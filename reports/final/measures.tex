\chapter{Game Measures}
\label{apdx:measures}

\section{Introduction}
These measures were developed based on the PhD thesis of Nicola Whitton \cite{whitton07}.
They were designed not to be used as interview questions, but instead as prompts to
the testers, to help them focus their observations on key things.

We planned to focus our observations on three areas: measuring the children's engagement
within the game, measuring the use of collaboration, and monitoring the reliability
of the software while in use. Although our measures are framed as a series of questions,
it was understood that interviewing children can be a difficult process, and
so the measures were used as guidelines for the testers, to guide their observations.

\subsection{Engagement}
In measuring the children's engagement within the game, we wanted to know specifically
how challenging they found it. While designing the game we planned to follow the optimal
game design curve where as we try to the keep the difficulty increasing at a linear rate.
We also looked to determine if the game concepts were easily understandable.

\subsection{Collaboration}
In measuring the use of collaboration within the game, we looked at how they communicated.
We decided we didn’t have time to add a chat system to the game and as such it was crucial to
see how much of an impact this would have on the players ability to communicate.
We also wanted to measure how effectively and easily they were able to combine operators to
form the the key number they needed to progress.

\subsection{System}
Lastly, we wanted to see how the software we had developed measured up under actual use,
to help us test the system and find areas for further development. It would also complement
the unit tests we have written for the software, to provide a fuller picture of the
software's reliability.

\section{Measures}
\subsection{Engagement with the game}

\begin{enumerate}
\item Measure average time it takes for a group of players to complete a level

\item Ask general questions about how challenging and immersive the game was
\begin{enumerate}
    \item Were there any points where they were confused or not sure what was going on?
    \item Were there aspects of the game that were too easy or too hard?
    \item How did the difficulty of the game change as you progressed through levels
\end{enumerate}

\item How much of game time is spent on task?
\begin{enumerate}
    \item What possible distractions are there?
\end{enumerate}

\item How often do they ask for help understanding the game concept?
\begin{enumerate}
    \item What sort of questions did they ask?
\end{enumerate}

\item Did they like the zombie characters?

\item What were their thoughts on the level maps?
\begin{enumerate}
    \item Were they interesting and varied?
\end{enumerate}

\end{enumerate}

\subsection{Use of collaboration}
\begin{enumerate}

\item How well did they communicate?

\item Did they understand the need to collaborate in order to advance?

\item Did they use their maths skills or did they randomly guess and match operators?

\end{enumerate}

\subsection{System integrity}
\begin{enumerate}

\item Did the game crash?

\begin{enumerate}
    \item What was the reason?
\end{enumerate}

\end{enumerate}
