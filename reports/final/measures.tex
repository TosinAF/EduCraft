\chapter{Game measures}
\label{apdx:measures}
These measures were developed based on the PhD thesis of Nicola Whitton \cite{whitton07}.
They were designed not to be used as interview questions, but instead as prompts to
the testers, to help them focus their observations on key things.

The use of these measures is described in Chapter \ref{ch:testing}.

\section{Engagement with the game}

\begin{enumerate}
\item Measure average time it takes for a group of players to complete a level

\item Ask general questions about how challenging and immersive the game was
\begin{enumerate}
    \item Were there any points where they were confused or not sure what was going on?
    \item Were there aspects of the game that were too easy or too hard?
    \item How did the difficulty of the game change as you progressed through levels
\end{enumerate}

\item How much of game time is spent on task?
\begin{enumerate}
    \item What possible distractions are there?
\end{enumerate}

\item How often do they ask for help understanding the game concept?
\begin{enumerate}
    \item What sort of questions did they ask?
\end{enumerate}

\item Did they like the zombie characters?

\item What were their thoughts on the level maps?
\begin{enumerate}
    \item Were they interesting and varied?
\end{enumerate}

\end{enumerate}

\section{Use of collaboration}
\begin{enumerate}

\item How well did they communicate?

\item Did they understand the need to collaborate in order to advance?

\item Did they use their maths skills or did they randomly guess and match operators?

\end{enumerate}

\section{System integrity}
\begin{enumerate}

\item Did the game crash?

\begin{enumerate}
    \item What was the reason?
\end{enumerate}

\end{enumerate}
