\chapter{Design}
\label{ch:design}
% report goes here

\section{Educational Design Decisions}

\section{Game Design}
The aim of our game design was to efficiently encode collaborative learning in the mechanics of the game. This section will list and describe all the additional elements that we decided to add to the existing Minecraft 1.6.4 core game in order to create a playable mod that would fulfill our requirements. We gradually introduced new elements as and when required.
\newline\newline
The mod is set to be played in adventure mode\footnote{Adventure mode is a game mode intended for player-created maps by limiting some of the gameplay in Minecraft, in which the player cannot directly destroy most blocks to avoid spoiling adventure maps or griefing servers. Most blocks cannot be destroyed without the proper tools. However, players can still interact with mobs and craft items.\cite{website:minecraft-adventure}}.
We have tried to preserve as much of the Minecraft concepts in our extension as possible, so that players familiar with the game can easily understand the mechanics of our mod. Inexperienced users will also get to know the Minecraft philosophy whilst using our extension and will be able to play the original game or some other mods without any problems in the future.

\begin{itemize}
\item \huge Items\large \footnote{Items are objects which only exist within the player's inventory and hands - which means, they cannot be placed in the game world. Some items simply place blocks or entities into the game world when used. They are thus an item when in the inventory and a block when placed.\cite{website:minecraft-item}}

\begin{itemize}

\Large \item Numbers
\newline
\normalsize These are the most important element in our extension and are crucial for promoting the practice of mathematics. Numbers are used in several of the features our mod offers.
\newline

\Large \item Operators
\newline
\normalsize Operators are as important as numbers, but only used in certain parts of the mod and are key elements of performing arithmetic operations.
\newline

\Large \item Keys
\newline
\normalsize Keys are used for progressing in the game. They allow us to place constraints on the free movement of players.
\newline

\Large \item Maths Wands
\newline
\normalsize Maths Wands enable us to limit the choice of weapons used for killing mobs.
\newline

\end{itemize}

\end{itemize}
