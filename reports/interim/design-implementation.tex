\chapter{Design and Implementation}

\section{Introduction}
In this chapter we present a discussion of the design of the system, 
and key factors that influenced our design. We will explain the requirement 
to develop using Java, and our decision to use the MinecraftForge API.

The online tutorials for making MinecraftForge mods informed our decision to 
create our mod following this system design; examples can be found on the
MinecraftForge wiki page \cite{website:mcforgewiki}.

\section{Key implementation decisions}

\subsection{Programming language}
Minecraft is a game written in the Java language, and so it is 
only logical that when we mod the game it is required that we use 
Java. This links in with the API we use, which is also dependent 
on Java and specifically is made with support for Eclipse (in 
terms for example of online tutorials for setting up and coding 
in the Eclipse JDK environment). 

\subsection{Application programming interface (API)}
Before creating our mod we had to choose the method for modding 
the game. Our final choice was to use the MinecraftForge API, which 
is a mod/application layer for Minecraft which lets us create and 
run Minecraft mods. The decision to use MinecraftForge API was 
because of its versatility and usefulness.

Furthermore, we decided 
to use MinecraftForge over other alternatives because, while other 
tools like ModLoader may have been arguably easier to use, MinecraftForge 
is much more suited for our purposes – namely, creating a more complex 
mod. This is because MinecraftForge has a lot more possibilities, whereas 
with ModLoader there is more limited scope for creating complex Minecraft 
mods. Aside from ModLoader and MinecraftForge, there wasn’t much else 
choice; these two mod tools are the most popular by far in the Minecraft 
community.  

\section{Initial design of the system}
Because of the very nature of MinecraftForge and our design, 
all modifications are done by creating new classes. Mods in Minecraft
are divided into two types: regular mods, which simply add new
functionality to the game, and `core mods' which alter some of the
game's existing functionality (changing gravity, for example).
Because our mod is a regular mod, our code is placed in a new package
separate from the rest of Minecraft's code, called \texttt{org.educraft}.

\subsection{Proxy classes}
All mods require proxy classses, which handle the rendering of entities.
Given the simple nature of our mod, our proxy classes are currently empty.

\subsection{Base class}
Mods made with MinecraftForge require a base class which provides instances
of all blocks and items which the mod adds into the game. For the dummy mod,
this is found in \texttt{org.educraft.DummyMod}.

\subsection{Item and entity classes}
All items, blocks, and entities that are added into the game extend the appropriate
pre-existing Minecraft classes: \texttt{net.minecraft.item.Item} in the case of
items and \texttt{net.minecraft.entity.Entity} in the case of blocks.

All of our custom items and entities were placed in the \texttt{org.educraft.dummy}
package. As far as possible, we chose to inherit from core classes which provided
the functionality we desired, to minimise the amount of new code that needed
to be written:
\begin{itemize}
    \item{\texttt{DummyCoin} and \texttt{DummyCoinPile} both inherit from 
            \texttt{net.minecraft.item.Item}, as they are simple items with no extra
            functions.}
    \item{\texttt{MathsWand} inherits from \texttt{net.minecraft.item.ItemSword},
            since it is an item that gets used as a weapon.}
    \item{\texttt{DummyZombie} inherits from \texttt{net.minecraft.entity.monster.EntityZombie},
            so that we can reuse the model and animation provided for the existing
            zombie entity.}
\end{itemize}

\subsection{Damage handling classes}
We also created two classes to handle the requirement that killing a Dummy Zombie
with the Dummy Sword / Maths Wand should cause a different item from normal
to be dropped. These were \texttt{DummyDamageSource}, which defined a new type
of damage in addition to the pre-existing types such as `fire', `physical', and
so on, and \texttt{DummyAttackHandler}, which provded a new event handler to
ensure that striking a zombie with the Dummy Sword inflicted the new damage type.

\section{Future development}
It is envisaged that the components of the completed mod will be heavily based on
the work done in the prototype. A new base class will need to be created, to define
the EduCraft mod (as opposed to the Dummy Mod), and some new items will need to
be created, but the basic principles will be carried over: we will extend
pre-existing classes wherever possible.

One new feature of the finished product that is not present in our current design
is custom texturing: for the dummy mod we used textures build into the game for
all of our items. This suffices for a prototype, but in the finished mod we will
need to provide our own textures to make it clear to players what items they
are using.